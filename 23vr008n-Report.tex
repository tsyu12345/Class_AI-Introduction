\documentclass{article}[jsarticle]
\usepackage[T1]{fontenc}
\usepackage[dvipdfmx]{hyperref}
\usepackage{lmodern}
\usepackage{latexsym}
\usepackage{amsfonts}
\usepackage{amssymb}
\usepackage{mathtools}
\usepackage{nccmath}
\usepackage{amsthm}
\usepackage{multirow}
\usepackage{graphicx}
\usepackage[dvipdfmx]{color}
\usepackage{wrapfig}
\usepackage{here}
\usepackage{float}
\usepackage{ascmac}
\usepackage{url}
\usepackage{pifont}


% Generated by ChatGPT
\usepackage{listings}
\usepackage{xcolor}

\lstset{
    basicstyle=\ttfamily\color{white},
    numbers=none,  % Line numbers
    numberstyle=\tiny\color{white},
    numbersep=5pt,
    tabsize=2,
    extendedchars=true,
    breaklines=true,
    keywordstyle=\color[rgb]{0.58,0.00,0.83},
    stringstyle=\color[rgb]{0.81,0.36,0.00},
    identifierstyle=\color{white},
    commentstyle=\color[rgb]{0.34,0.62,0.16},
    rulecolor=\color[rgb]{0.5,0.5,0.5},
    xleftmargin=0.1cm,    % Left margin
    xrightmargin=0.1cm,   % Right margin
    language=python,
    backgroundcolor=\color[rgb]{0.13,0.13,0.13},
    showspaces=false,
    showstringspaces=false
}


\title{人工知能概論 課題レポート}
\author{高林秀 \\ 三宅研究室 博士前期課程1年 \\ V-CampusID : 23vr008n}
\date{\today}

\begin{document}

\maketitle

\begin{abstract}
    本稿は本年度の人工知能概論の課題レポートである。
    各課題は\ding{172}~\ding{174}の3つに分かれており、以下各セクションでそれぞれの課題について説明、および解答するものとする。
\end{abstract}

\section{課題\ding{172}}
\subsection{課題内容}
\begin{enumerate}
    \item 人工知能の定義を自分なりにしてください。(200-500字)
    \item (1) の裏付けとなる根拠を述べてください。少なくとも、3つ以上の書籍か論文から引用してください。
    \item(1)とまったく逆の人工知能の定義を自分なりにしてください。(200-500字)。
    \item (3) の裏付けとなる根拠を述べてください。少なくとも、3つ以上の書籍か論文から引用してください。
\end{enumerate}

\section{課題\ding{173}}
\subsection{課題内容}
\begin{enumerate}
    \item 社会、或いは個人的に必要だと思う新しい人工知能を一つ上げてください。その役割を述べてください(200-500字)
    \item 上記の人工知能をエージェント・アーキテクチャをベースに設計し、解説してください。少なくとも、一つのエージェント・アーキテクチャ図を描いてください。特に、次の3つの点は必ず書いてください。
    (300-700字)
    \begin{itemize}
        \item センサリングするデータ
        \item 思考の形(データの抽象化・変形・意思決定など)
        \item アウトプットするアクション
    \end{itemize}
\end{enumerate}
    
\section{課題\ding{174}}
\subsection{課題内容}
\begin{enumerate}
    \item スマートシティの全体のアーキテクチャについて設計をしてください。
    具体的にエージェント・アーキテクチャ図を描いてください。
    特に以下の点を含んでください。
    \begin{itemize}
        \item 都市全体を以下に認識するか
        \item 都市全体をロボットやドローン、バーチャルキャラクターを使っていかにコントロールするか。
        \item 都市全体の人工知能はいかに意思決定を行うか。
    \end{itemize}
    \item (1) で作ったアーキテクチャはいかに人々の都市生活を改善するか、
    を述べてください。(400-1000字)
    \item スマートシティとメタバースの関係を述べてください。(400-1000字)
\end{enumerate}


\end{document}
